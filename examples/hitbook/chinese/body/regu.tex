% !Mode:: "TeX:UTF-8"

\chapter[课题的来源及研究的目的和意义]{课题的来源及研究的目的和意义}[Harbin Institute of Technology Postgraduate Dissertation Writing Specifications]


\section{课题来源}[Content specification]
%\subsection{题目}[Title]

国家重点研发计划“基于国产卫星的历史数据再构建研究”。

\section{研究的目的和意义}[Abstraction and key words]
%\subsubsection{摘要}[Abstraction]

强对流天气指的是突然发生、强度剧烈,常伴有短时强降水、雷电、大风、冰雹、龙卷等强烈对流性灾害的天气。强对流天气空间尺度不大、生命周期短暂,却能在短时间内释放强大的力量,具有极强的破坏力,是人们不得不预防的灾害性天气。强对流天气时效短暂,危害大,其产生的强降雨往往在较短时间内就可以达到暴雨的量级,是引发城市内涝、山洪爆发以及山体滑坡和泥石流之类的地质灾害的主要源头。其中,短时大风可以达到8级以上,破坏树木或房屋;雷电则对电力电信设施和生命财产构成威胁,引发雷击火险和森林火灾;冰雹从5毫米到10厘米大小不等,最大的可以达到30厘米,可直接摧毁地里的庄稼。

强对流是空气强烈的垂直运动而导致出的天气现象。强对流的形成主要是因为云团受到的太阳辐射导致的云团热胀,密度减小,从而发生向上的垂直运动,在运动过程中气温逐渐降低,空气中包含的水蒸气凝结成为水滴,水滴在下降过程中又会被其他上升气流携升,如此反复最后积累成大水滴直至高空气流无力支持其重量,最后下降成雨。

预防强对流天气是天气预报的重要课题,也关乎人民人身财产安全。对于预防强对流天气,通常采用的方法是采取精度更高的预报措施,对于已经形成了的对流进行跟踪预测,对于还在形成的对流云进行判别分析。早期的强对流天气预警主要是基于雷达资料,初生对流的定义是在强对流天气发生之前,大气各种物理量处于某种特殊的状态,多普勒天气雷达第一次观测到对流云产生的反射率因子大于等于35dBZ[1]。雷达图像的生成主要依靠雷达回波产生,这就导致只有具有雷达观测站的区域才有能够采集的雷达数据。单纯依靠雷达数据做对流研究预警存在两个问题,一是雷达站的回波区域并不能做到对于全国的全覆盖,因此对于全局对流的形成和移动不能做到最为合理和及时的判断;二是雷达数据本身的局限性,导致其信息不足以对初生对流进行合理判断。

因此我们可以利用卫星云图对对流云进行进一步预测分析。卫星云图具有长时效性,范围广等特点。强对流天气是以大尺度天气系统为背景,大尺度天气系统影响或决定着中小尺度天气系统的生成、发展和移动过程。利用卫星云图可以有效监控大尺度天气系统,从而对初生对流进行有效预测。

强对流天气严重威胁人民生命财产安全,如果能够有效监测初生对流,那么将会大幅提高对流的预警时效,减少强对流天气带来的损失。

随着计算机算力的提升,利用深度学习结合卫星数据对于解决初生对流的检测的问题带来了新的解决方案。卫星数据具有很长的时序性,数据量充足,国产静止卫星监控的范围辽阔,分辨率高,足够作深度学习的数据集来源。除此之外,我国卫星数据还具有较多的历史信息,通过对于历史数据的信息的充分挖掘,可以得到对流的初生、移动、变化、消亡等特征。因此可以针对卫星数据做时序图像检测或预测。

对卫星数据做初生对流检测研究,初生对流检测与强对流检测不同,此时的云团虽然内部有一定变化,但是利用通常的阈值法不能很好的进行检测,对流云在检测过程中也会出现将卷云误判为对流的情况,初生对流表现在卫星图像上利用传统的方法很难取得很好的效果。所以要依靠深度学习的方法深度挖掘对流在形成前的云的特征,在对流形成之前将初生对流检测出来。有对初生对流的基本判断下,可以达到更为及时的强对流预警。

在实现初生对流的检测,需要对卫星云图做最基本的对流的检测及预测,卫星云图本身就带有对流的基本信息,可以通过对流目标检测的方式将对流检测出来,通过上一帧中的对流信息及图形信息,预测下一帧中的对流信息,从而得到这一帧中的初生对流可能存在的区域。


%\subsubsection{关键词}[Keywords]
%关键词是供检索用的主题词条。关键词应集中体现论文特色,反映研究成果的内涵,具有语
%义性,在论文中有明确的出处,并应尽量采用《汉语主题词表》或各专业主题词表提供的规
%范词,应列取3$\sim$6个关键词,按词条的外延层次从大到小排列。
%
%\subsection{目录}[Content]
%
%论文中各章节的顺序排列表,包括论文中全部章、节、条三级标题及其页码。
%
%\subsection{论文正文}[Main body]
%
%论文正文包括绪论、论文主体及结论等部分。
%
%\subsubsection{绪论}
%绪论一般作为第1章。绪论应包括:本研究课题的来源、背景及其理论意义与实际意义;国
%内外与课题相关研究领域的研究进展及成果、存在的不足或有待深入研究的问题,归纳出将
%要开展研究的理论分析框架、研究内容、研究程序和方法。
%
%绪论部分要注意对论文所引用国内外文献的准确标注。绪论的主要研究内容的撰写宜使用将
%来时态,切忌将论文目录直接作为研究内容。
%
%\subsubsection{论文主体}
%论文主体是学位论文的主要部分,应该结构严谨,层次清晰,重点突出,文字简练、通顺。
%论文各章之间应该前后关联,构成一个有机的整体。论文给出的数据必须真实可靠,推理正
%确,结论明确,无概念性和科学性错误。对于科学实验、计算机仿真的条件、实验过程、仿
%真过程等需加以叙述,避免直接给出结果、曲线和结论。引用他人研究成果或采用他人成说
%时,应注明出处,不得将其与本人提出的理论分析混淆在一起。
%
%论文主体各章后应有一节“本章小结”,实验方法或材料等章节可不写“本章小结”。各章
%小结是对各章研究内容、方法与成果的简洁准确的总结与概括,也是论文最后结论的依据。
%
%\subsubsection{结论}
%结论作为学位论文正文的组成部分,单独排写,不加章标题序号,不标注




% Local Variables:
% TeX-master: "../main"
% TeX-engine: xetex
% End: